%! Author = Runge
%! Date = 29-12-2023

% Packages
\RequirePackage{clrscode4e}

\usepackage{pgfplots}
\pgfplotsset{compat=1.18}
\usepackage{microtype}
\usepackage{stix2}
\usepackage{mathtools}
\usepackage{booktabs}
\usepackage{tikz}
\usepackage{amsthm}
\usepackage{thm-restate}
\usepackage{multirow}
\usepackage{bbm}

% Packages with options set
\usepackage[hidelinks]{hyperref}
\usepackage[textsize=small,obeyDraft]{todonotes}
\usepackage[newfloat]{minted}
\usepackage[backend=biber,
      bibencoding=utf8,
      maxbibnames=20,
      style=ieee,
      citestyle=numeric-comp,
      url=false
]{biblatex}
\usepackage[acronym]{glossaries}
\newcommand{\ceil}[1]{\left\lceil #1 \right\rceil}

% Hyperlink and PDF properties
\usepackage{orcidlink}
\makeatletter
\hypersetup{%
      plainpages=false,%
      pdftitle=\@title,%
      pdfauthor={Sebastian Aaholm, Lars Emanuel Hansen, Daniel Runge Petersen},%
      pdflang={en-GB},%
      pdfsubject={Semester project at Aalborg University},%
      pdfkeywords ={Formal Verification, Parameter Estimation, Decision Diagram}
      bookmarksnumbered=true,%
      colorlinks=true,%
      citecolor=black,%
      filecolor=black,%
      linkcolor=black,% you should probably change this to black before printing
      urlcolor=black,%
      pdfstartview=FitH%
}
\makeatother

% Package setup
\setlength{\marginparwidth}{2cm} % todonotes width
\setminted{linenos=true, autogobble, breaklines, fontsize=\footnotesize, style=friendly, xleftmargin=1em, numbersep=5pt, frame=lines}
\addbibresource{bib/main.bib}

% Other setup and options
\usetikzlibrary{shapes.geometric, arrows, fit, calc, automata, positioning, tikzmark,chains, scopes}
%%% done
% Events
\usetikzlibrary{shapes.multipart}
\tikzstyle{arrow} = [thick,->,>=stealth]

\tikzstyle{state} = [rectangle, minimum width=2cm, minimum height=0.7cm, text centered, draw=black, fill=white!30, align=center]
\tikzstyle{cluster} = [line width=0.4pt, draw=black, inner sep=0.5em, rounded corners=0.1cm]
\declaretheorem{theorem}
\declaretheorem{lemma}
\declaretheorem{corollary}
\declaretheorem{definition}
\declaretheorem{example}
\declaretheorem{conjecture}
\DeclareFloatingEnvironment[name=Algorithm, placement=htbp]{algorithm}
\floatplacement{listing}{htbp}
\floatplacement{table}{tbhp}
% \floatplacement{figure}{tbhp}


\newcommand{\Jajapy}{\textsc{Jajapy}}
\newcommand{\Cupaal}{\textsc{CuPAAL}}
\newcommand{\JajapyTwo}{\textsc{Jajapy 2}}
\newcommand{\Storm}{\textsc{Storm}}
\newcommand{\Prism}{\textsc{Prism}}
\newcommand{\Stormpy}{\textsc{Stormpy}}
\makeglossaries
