\documentclass[a4paper]{article}

%! Author = Runge
%! Date = 29-12-2023

% Packages
\RequirePackage{setup/clrscode4e}
\usepackage{microtype}
\usepackage{mathtools, thm-restate}
\usepackage{stix2}
\usepackage{amsmath, amsthm}
\usepackage{booktabs}
\usepackage{tikz}

% Packages with options set
\usepackage[hidelinks]{hyperref}
\usepackage[textsize=small,obeyDraft]{todonotes}
\usepackage[newfloat, outputdir=../out]{minted}
\usepackage[backend=biber,
    bibencoding=utf8,
    maxbibnames=20,
    style=ieee,
    citestyle=numeric-comp,
    url=false
]{biblatex}
\usepackage[acronym]{glossaries}

% Hyperlink and PDF properties
\usepackage{orcidlink}
\makeatletter
\hypersetup{%
    plainpages=false,%
    pdftitle=\@title,%
    pdfauthor={Sebastian Aaholm, Lars Emanuel Hansen, Daniel Runge Petersen},%
    pdflang={en-GB},%
    pdfsubject={Semester project at Aalborg University},%
    pdfkeywords ={Formal Verification, Parameter Estimation, Decision Diagram}
    bookmarksnumbered=true,%
    colorlinks=true,%
    citecolor=black,%
    filecolor=black,%
    linkcolor=black,% you should probably change this to black before printing
    urlcolor=black,%
    pdfstartview=FitH%
}
\makeatother

% Package setup
\setlength{\marginparwidth}{2cm} % todonotes width
\setminted{linenos, autogobble, breaklines, fontsize=\footnotesize, style=friendly, xleftmargin=1em, numbersep=5pt}
\addbibresource{bib/main.bib}

% Other setup and options
\declaretheorem{theorem}
\declaretheorem{lemma}
\declaretheorem{corollary}
\declaretheorem{definition}
\declaretheorem{example}
\declaretheorem{conjecture}
\DeclareFloatingEnvironment[name=Algorithm, placement=htb!]{algorithm}

\makeglossaries

%! Author = Runge
%! Date = 29-12-2023

\newacronym{aau}{AAU}{Aalborg University}
\newacronym{mc}{MC}{Markov Chain}
\newacronym{hmm}{HMM}{Hidden Markov Model}
\newacronym{mdp}{MDP}{Markov Decision Process}
\newacronym{ctmc}{CTMC}{Continuous Time Markov Chain}
\newacronym{dtmc}{DTMC}{Discrete Time Markov Chain}
\newacronym{bw}{BW}{Baum-Welch}
\newacronym{add}{ADD}{Algebraic Decision Diagram}
\newacronym{bdd}{BDD}{Binary Decision Diagram}
\newacronym{pctmc}{pCTMC}{Parametric Continuous Time Markov Chain}
\newacronym{cudd}{CuDD}{Colorado University Decision Diagram}
\newacronym{cthmm}{CTHMM}{Continuous Time Hidden Markov Model}
\newacronym{iid}{i.i.d.}{independently identically distributed}
\newacronym{sul}{SUL}{System Under Learning}
\newacronym{nand}{NAND}{NAND multiplexing}
\newacronym{brp}{BRP}{Bounded Retransmission Protocol}


\title{Summary of the paper: Symbolic Baum-Welch Algorithm for Learning Probabilistic Models with Algebraic Decision Diagrams}
\date{\today}
\author{Daniel Runge Petersen \and Lars Emanuel Hansen \and Sebastian Aaholm}
\pagestyle{empty}
\begin{document}
\maketitle
\thispagestyle{empty}
This thesis introduces a symbolic approach to the \gls{bw} algorithm using \glspl{add} for efficiently learning parameters of \glspl{hmm} and \glspl{mc}.
Traditional recursive or matrix-based implementations of \gls{bw} often suffer from high memory usage and limited scalability.
To address this, we propose a novel symbolic implementation built into a C++ library named \Cupaal, which is integrated into the Python-based \Jajapy\ library as \JajapyTwo.

The \gls{bw} algorithm is a form of Expectation-Maximization (EM) algorithm commonly used to estimate parameters of \glspl{hmm} from observed sequences.
Tools like \Jajapy\ implement \gls{bw} recursively or through matrix operations.
However, these approaches become inefficient for large or repetitive models due to memory constraints and computational redundancy.

The motivation behind \Cupaal\ is to overcome these issues by leveraging \glspl{add}, data structures that compactly represent numerical functions over discrete variables.
\glspl{add} generalize \glspl{bdd}, allowing them to represent and manipulate probabilities and other numeric values symbolically.
This symbolic computation reduces redundancy and memory use, making it feasible to train large-scale probabilistic models.

\Cupaal represents transition, emission, and initial state matrices as \glspl{add}.
Matrix operations, including the Kronecker and Hadamard products, are symbolically implemented.
This structure is especially beneficial when matrix sparsity or redundancy exists.
PRISM models are translated into \Jajapy models and then encoded as \glspl{add} for use in \Cupaal.

The main contributions of this work are:
\begin{enumerate}
    \item Symbolic Reformulation of the \gls{bw} Algorithm: Each step of the \gls{bw} algorithm (forward-backward, parameter updates) is redefined in terms of \gls{add} operations using the \gls{cudd} library.
    \item Multi-Sequence Learning Support: The symbolic implementation handles multiple observation sequences for both \glspl{mc} and \glspl{hmm}.
    \item Integration with \Jajapy: The symbolic implementation is integrated into the \Jajapy library, allowing users to switch between traditional and symbolic learning modes.
    \item Empirical Evaluation: Using models from the QComp benchmark (specifically, the leader sync model), experiments demonstrate:
          \begin{itemize}
              \item Comparable or better runtime performance, especially for models with repeated structural patterns.
              \item Accuracy on par with traditional methods.
              \item Improved scalability, especially when models are initialized with repeated values (semi-random initialization).
          \end{itemize}
\end{enumerate}

% The experiments address three research questions:
% \begin{itemize}
%     \item Scalability: \Cupaal\ performs significantly better than \Jajapy\ for large models and long observation sequences, especially when observation redundancy can be exploited.
%     \item Accuracy: Both implementations show nearly identical accuracy, with minor variations in log-likelihood and transition probabilities.
%     \item Initialization Impact: Semi-random initialization (fewer unique values) improves \Cupaal's performance due to its reliance on structure reusability in \glspl{add}.
% \end{itemize}

The symbolic implementation of the \gls{bw} algorithm using \glspl{add} proves to be a scalable and efficient alternative to traditional approaches.
With \JajapyTwo, users now have access to a flexible tool for probabilistic model learning that maintains accuracy while improving performance for complex and large-scale models.
\end{document}