\section{Improvements}\label{sec:improvements}
This section outlines the improvements gained by transitioning from the recursive implementation in Jajapy to the symbolic approach in Cupaal.

As discussed in (Ref to previous section talking about Jajapy), Jajapy uses a recursive implementation of the Baum-Welch algorithm to learn \glspl{hmm}.
In contrast, Cupaal implements the Baum-Welch algorithm using \glspl{add}.
By leveraging \glspl{add}, Cupaal demonstrates significant improvements over the recursive approach used in Jajapy.
The discussion of these improvements is based on the experimental results presented in Section~\ref{sec:experiments}.

\subsection{Run Time}\label{subsec:improvements_run_time}
One of the most notable advantages of Cupaal over Jajapy is the reduction in run time, particularly for models with a large number of states.

The use of \glspl{add} minimizes redundant computations by merging identical values within the structure.
This optimization significantly reduces the computational needs, compared to a recursive implementation.
As a result, the run time gap between Cupaal and Jajapy increases as the number of states grows, making Cupaal a more scalable solution for large \glspl{hmm}.

The number of iterations of Cupaal for each model is also slightly reduced compared to Jajapy.

This advantage is particularly beneficial in scenarios where handling large probability matrices would otherwise lead to excessive computational costs.


\subsection{Accuracy}\label{subsec:improvements_accuracy}
While run time is a key advantage of Cupaal, it is equally important to assess whether these performance gains come at the cost of accuracy.
Since both approaches implement the Baum-Welch algorithm, they are expected to converge to similar model parameters when learning \glspl{hmm}.

As shown in Section~\ref{sec:experiments}, Cupaal achieves accuracy comparable to Jajapy across various models.
These results can be seen in the values of avg delta and the log-likelihood, where the closer the value is to 0 the better.
Displaying that a symbolic implementation does not introduce significant numerical errors, ensuring that the learned transition and emission probabilities remain consistent with those obtained using the recursive approach.

Furthermore, by eliminating redundant calculations, Cupaal may reduce floating-point errors that typically accumulate in recursive implementations.
Importantly, Cupaal maintains accuracy even as the number of states increases, showcasing that its efficiency improvements do not compromise learning quality.
This makes it particularly well-suited for handling large-scale \glspl{hmm}.


\subsection{Implementation}\label{subsec:improvements_implementation}
The implementation of Cupaal has been done in c++, compared to Jajapy which is implemented in Python.
This could also be a factor aiding the performance improvement of Cupaal, as C++ is generally faster at computation compared to Python.
This choice of implementation not only improves speed but also ensures that Cupaal can efficiently handle large models that would be infeasible in Python.

\subsection{Final improvement overview}\label{improvements_overview}
The improvements introduced by Cupaal stem from multiple factors: the adoption of \glspl{add}, optimized run time and a high-performance C++ implementation.
These enhancements make Cupaal a powerful alternative to recursive approaches like Jajapy, particularly when working with large, redundant, and complex \glspl{hmm}.





