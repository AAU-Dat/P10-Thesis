\section{Previous Work}\label{sec:jajapy_and_cupaal}
In this section, we provide a brief overview of previous work that has influenced our research and has been iterated upon.
Specifically, we discuss what these tools are, how they function, who utilizes them, and the motivations behind integrating them into our research.
The focus will be on two primary tools: Jajapy and CuPAAL.


\subsection{Jajapy}\label{subsec:jajapy}
Jajapy is a Python library developed to implement the Baum-Welch algorithm for learning various types of Markov models.
The library employs a recursive implementation of the Baum-Welch algorithm to infer the parameters of these models.

Jajapy was chosen for our research due to its established use of recursive methods in training Hidden Markov Models (HMMs) and other probabilistic models.
Our objective was to explore the impact of using \glspl{add} in the Baum-Welch algorithm, as opposed to conventional matrix-based or recursive methods.
Jajapy served as an appropriate baseline for comparison, allowing us to evaluate the potential computational efficiency and accuracy improvements that \glspl{add} might offer.


\subsection{CuPAAL}\label{subsec:cupaal}
CuPAAL is a tool developed in C++ that extends the work done in Jajapy by implementing the Baum-Welch algorithm with an \gls{add}-based approach instead of a recursive method.
The goal of CuPAAL is to leverage \glspl{add} to improve the efficiency of learning Markov models, particularly in large-scale applications where traditional recursive methods may become computationally expensive.

CuPAAL has undergone multiple iterations. Initially, it implemented a partial \gls{add}-based approach, where only certain components of the Baum-Welch algorithm were optimized using \glspl{add}.
This partial implementation served as an initial proof-of-concept to determine whether incorporating \glspl{add} could yield performance benefits compared to the recursive approach employed by Jajapy.

Following promising results from the partial implementation, further development led to a fully \gls{add}-based version of CuPAAL.
This iteration replaced all recursive computations with \glspl{add}, enabling more efficient execution, particularly for large models.
The transition to a fully \gls{add}-based approach demonstrated the potential for significant computational savings and scalability improvements, reinforcing the viability of this method for broader applications beyond our initial research scope.

By building upon Jajapy and developing CuPAAL, we have been able to evaluate the impact of using \glspl{add} in probabilistic model learning.
