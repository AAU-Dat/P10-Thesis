\section{Implementation}\label{sec:implementation}
This section provides an overview of \Cupaal's implementation, including the \gls{bw} algorithm, and how \Cupaal\ is integrated into \Jajapy, creating \JajapyTwo.

\subsection{Motivation for CuPAAL}\label{subsec:motivation-for-cupaal}
The motivation for \Cupaal\ is to provide a more efficient and scalable implementation of the \gls{bw} algorithm for parameter estimation.
Specifically, we aim to improve the performance of the algorithm when handling large and complex models, and address the existing limitations of the \gls{bw} algorithm in \Jajapy.

\subsubsection{Recursive vs. Matrix vs. ADD-based Approaches}\label{subsec:approaches}
When working with the \gls{bw} algorithm, different approaches can be taken to optimize computational efficiency.
Three common strategies are recursive, matrix-based, and \gls{add}-based approaches, each with distinct advantages and limitations.


\begin{itemize}
    \item \textbf{Recursive Approach:} Conceptually simple, recursion follows a divide-and-conquer strategy and makes use of a dynamic programming approach. Previous calculations are used to build upon future calculations. These results are stored in a list or a map, allowing them to be accessed when needed~\cite[Chapter 4]{cormen2022introduction}.
    \item \textbf{Matrix Representation:} Reformulating algorithms using matrix operations leverages algebraic properties for parallel computation and efficient processing.
          By building upon the recursive approach, matrices provide an efficient method for accessing stored results, leading to faster computations overall~\cite[Chapter 4, 15 \& 28]{cormen2022introduction}.
    \item \textbf{ADD-based Approach:} \glspl{add} provide a compact representation that eliminates redundancy in recursive computations.
          By reusing previously computed substructures, they improve efficiency and reduce memory overhead~\cite{bahar1997algebric}.
          Compared to matrices, \glspl{add} can offer a more space-efficient alternative for structured data while extending \gls{bdd} techniques to handle both Boolean and numerical computations.
\end{itemize}


In this work, we investigate the advantages of \gls{add}-based approaches for solving complex problems, with a focus on parameter estimation in \gls{mc} and \gls{hmm}. We compare the performance of \gls{add}-based algorithms against recursive-based implementations, highlighting the advantages of using \glspl{add} for efficient computation and memory management.

\subsection{What is CuPAAL}\label{subsec:what_is_cupaal}
\Cupaal\ is a C++ library that implements the \gls{bw} algorithm for parameter estimation, which has evolved over time.

The initial version of \Cupaal\ was written in C and called \textit{SUDD}, a partial implementation of the \gls{bw} algorithm that utilized \glspl{add}. This version was primarily focused on demonstrating the efficiency of \glspl{add} for parameter estimation problems and was not fully functional. The next iteration was called \Cupaal, which was a complete implementation of the \gls{bw} algorithm using \glspl{add}. However, it only supported \glspl{hmm} and was only designed to make use of a single observation.

The current version of \Cupaal\ has been extended to support \glspl{mc} and can handle multiple observations. This version of \Cupaal\ is designed for standalone use, as well as in conjunction with \Jajapy, facilitating easy integration and application in parameter estimation problems.

The following sections provide an overview of what \Cupaal\ is and its capabilities.

\subsubsection{What Does Cupaal Contain}\label{subsec:what_does_cupaal_contain}
Throughout all its iterations, \Cupaal\ has utilized the \gls{cudd} library - a library for implementing and manipulating \glspl{bdd} and \glspl{add} developed at the University of Colorado.

Implemented in C, the \gls{cudd} library ensures high-performance execution and can be seamlessly integrated into C++ programs, which we utilize in \Cupaal.
By leveraging the \gls{cudd} library, we demonstrate the benefits of \gls{add}-based approaches for solving parameter estimation problems in \glspl{mc}.

The \gls{cudd} library is used to store \glspl{add} and perform operations on them.
Its optimized algorithms and efficient memory management enable symbolic handling of large and complex matrices, significantly improving performance compared to traditional methods.

We have not modified or extended the \gls{cudd} library directly, but we have added functions that wrap several functions of \gls{cudd}.
All functionality used in our implementation is available through the standard \gls{cudd} library.

\subsubsection{From Prism to Cupaal}\label{subsec:from_prism_to_cupaal}
In the current iteration of \Cupaal, it is possible to use \Prism\ models as input to the \gls{bw} algorithm. The models are encoded from \Prism\ models to \Cupaal\ models, which is achieved by parsing the \Prism\ model into \Jajapy\ using \Stormpy.

The Jajapy model comprises a transition matrix, a label matrix, and an initial state vector.
The model is passed to \Cupaal, where these matrices and vectors are encoded into \glspl{add} as a function $f \colon \{0,1\}^n \times \{0,1\}^n \to R$.

The Transition matrix is a $S\times S$ matrix, where $S=States$, and is encoded to an \gls{add} by each row and column with a binary value. This value is determined based on the size of the matrix,
$n = \ceil{log_2(S)}$.

The label matrix is a $S\times L$ matrix, where $L=Labels$, and since there is no guarantee that $S = L$, the encoding is handled differently.
The matrix is instead treated as a list of vectors.

Each vector is encoded as square matrices, where each row or column (depending on the vector type) is duplicated, which is then encoded to a list of \glspl{add}.

Knowing the exact dimensions of matrices and that they are square helps to simplify some of the symbolic operations.
An example of this is provided in \autoref{subsec:kronecker-product-implementation}.

The Initial state vector is encoded similarly to the label matrix, but only as a single \gls{add}.

\subsubsection{Kronecker Product Implementation}\label{subsec:kronecker-product-implementation}
The Kronecker product is implemented in \Cupaal\ using the row and column duplication method mentioned in \autoref{subsec:from_prism_to_cupaal}.

The structure of Decision Diagrams in \Cupaal, where keeping track of all the new binary values used for encoding from a matrix to an \gls{add}, can add a layer of complexity for calculation.
Especially when computing operations that translate matrices to new dimensions, such as the Kronecker product.

Here we present a variation of the Kronecker product that only works between vectors - specifically one row and one column vector, as it relies on the structure of the vectors being expanded into square matrices.

This matrix-based approach enables efficient symbolic operations, as the Kronecker product can be calculated by taking the Hadamard product between a column matrix \gls{add} and a row matrix \gls{add}, simplifying what would otherwise be a more complex operation.

An example of this can be seen with the two vectors $\hat{A}$ and $\hat{B}$:

Let $\hat{A} = \begin{bmatrix}
        1 \\
        2
    \end{bmatrix}$
and $\hat{B}=\begin{bmatrix}
        3 & 4
    \end{bmatrix}$.

The Kronecker product of these two vectors is computed as follows:

\begin{equation}
    \hat{A} \otimes \hat{B} = \begin{bmatrix}
        1 \cdot 3 & 1 \cdot 4 \\
        2 \cdot 3 & 2 \cdot 4
    \end{bmatrix} = \begin{bmatrix}
        3 & 4 \\
        6 & 8
    \end{bmatrix}.
    \label{eq:kronecker-product-example}
\end{equation}

Another way to calculate the Kronecker product is to expand the vectors into matrices.
$\hat{A}$ and $\hat{B}$ are expanded to be matrices, similar to how the matrix was treated as a list of vectors and then expanded to square matrices, as seen with the Label matrix.

Let $\mathbf{A} = \begin{bmatrix}
        1 & 1 \\
        2 & 2
    \end{bmatrix}$ and
$\mathbf{B} = \begin{bmatrix}
        3 & 4 \\
        3 & 4
    \end{bmatrix}$.

The Kronecker product of $\hat{A}$ and $\hat{B}$ can also be calculated, by using the Hadamard product of $\mathbf{A}$ and $\mathbf{B}$.
This is done as follows:


\begin{equation}
    \mathbf{A} \odot \mathbf{B} = \begin{bmatrix}
        1 \cdot 3 & 1 \cdot 4 \\
        2 \cdot 3 & 2 \cdot 4
    \end{bmatrix} = \begin{bmatrix}
        3 & 4 \\
        6 & 8
    \end{bmatrix}.
\end{equation}


Hereby showing that the Hadamard product can be used to compute the Kronecker product between two vectors, by using the row and column duplication method.

\subsection{Implementation to Jajapy}\label{subsec:implementation-to-jajapy}
This section provides an overview of how \Cupaal\ is implemented in Jajapy, utilizing bindings between C++ and Python. \autoref{fig:cupaal-jajapy-architecture} shows the overall architecture of the implementation.


\begin{figure}
    \centering
    \usetikzlibrary{shapes.geometric, arrows, fit, calc, automata, positioning, tikzmark,chains, scopes}
%%% done
% Events
\usetikzlibrary{shapes.multipart}
\tikzstyle{arrow} = [thick,->,>=stealth]

\tikzstyle{state} = [rectangle, minimum width=2cm, minimum height=0.7cm, text centered, draw=black, fill=white!30, align=center]
\tikzstyle{cluster} = [line width=0.4pt, draw=black, inner sep=0.5em, rounded corners=0.1cm]
\begin{tikzpicture}[start chain=1 going below,
        start chain=2 going below,
        node distance=.5cm and 1.5cm,
        roundnode/.style={circle, minimum size=7mm, text centered, draw=black, fill=white!30, align=center},
        every node/.style={font=\small}]

    \node [on chain=1, state] (trainingset) {main.cpp} node[below,scale=1, xshift=-15,yshift=30]{CuPAAL};
    \node [on chain=1, state, fill=green] (learning) {BW.cpp};
    \node [on chain=1, state, fill=green] (bind) {bindings.cpp};

    \node [on chain=2, roundnode, right=of trainingset] (cupaal) {cupaal.exe};
    \node [on chain=2, roundnode, fill=green] (bindings) {bindings.so};
    \node [below right=of cupaal, state] (candice) {fit} node[below,scale=1, xshift=165,yshift=-15]{Jajapy};

    \node[draw, dashed, fit={(trainingset) (learning) (bind)},
        inner xsep=10pt, inner ysep=20pt] (box1) {};
    \node[draw, dashed, fit={(candice)},
        inner xsep=10pt, inner ysep=15pt] (box1) {};
    \coordinate (Qf) at ([xshift=-0.8cm, yshift=-0.5cm]cupaal.west);
    \coordinate (Qg) at ([xshift=-0.8cm, yshift=0.5cm]bindings.west);

    \draw[arrow, -] (trainingset.east) --  (Qf);
    \draw[arrow, -] (learning.east) --  (Qf);
    \draw[arrow, -] (bind.east) --  (Qg);
    \draw[arrow, -] (learning.east) --  (Qg);
    \draw[arrow, ->] (Qf) -- (cupaal.west);
    \draw[arrow, ->] (Qg) -- (bindings.west);
    \draw[arrow, ->] (bindings.east) -- (candice.west);
\end{tikzpicture}
    \caption{Architecture of \Cupaal\ combined with \Jajapy.}
    \label{fig:cupaal-jajapy-architecture}
\end{figure}

\Cupaal\ consists of two main components: the main function and the \gls{bw} library. Both of these are compiled into an executable program called \texttt{cupaal.exe}, which can be used to run the \gls{bw} algorithm on a given model.

\subsubsection{Bindings}\label{subsubsec:bindings}
To implement \Cupaal\ into \Jajapy, we create bindings between C++ and Python using the \texttt{pybind11} library~\cite{pybind11github}, which allows us to call C++ functions from Python, enabling us to use \Cupaal\ in \Jajapy.

In the code examples, some parts have been removed for brevity and clarity.


\begin{listing}
    \begin{minted}{cpp}
// Some parameters have been omitted for brevity
cupaal_markov_model cupaal_bw_symbolic(vector<string>& states, vector<string>& labels, vector<vector<string>>& observations, vector<double>& initial_distribution, vector<double>& transitions, vector<double>& emissions, int max_iterations = 100, double epsilon = 1e-2){
    MarkovModel model(states, labels, initial_distribution, transitions, emissions, observations);
    cupaal_markov_model model_data;
    chrono::seconds time = chrono::seconds(3600);
    model.baum_welch_multiple_observations(
        max_iterations, epsilon, time);

    // output and result path omitted for brevity
    model_data.initial_distribution = model.initial_distribution;
    model_data.transitions = model.transitions;
    model_data.emissions = model.emissions;
    
    Cudd_Quit(model.manager);
    return model_data;
}
      \end{minted}
    \caption{C++ bindings file for CuPAAL}
    \label{lst:cupaal-bindings}
\end{listing}

We create a C++ bindings file that uses the \gls{bw} library from \Cupaal\ and define the function we want to expose to Python; we call this function $cupaal\_bw\_symbolic$, seen in~\autoref{lst:cupaal-bindings}.
This function takes model parameters from a \Jajapy\ model as input and transforms them for use in \Cupaal.
The transformation is done at line 3, where all the parameters are inputted to create a \texttt{Markov Model} object, which is then used to run the \gls{bw} algorithm at line 6.

Each of the values relevant to the \gls{bw} algorithm is then passed into the \texttt{model\_data} object, which is subsequently returned to \Jajapy, as seen in lines 10 through 15. These are the initial distribution, the transitions and the emissions.


\begin{listing}
    \begin{minted}{cpp}
        void baum_welch_multiple_observations(
            unsigned int max_iterations = 100, 
            double epsilon = 1e-6, 
            chrono::seconds time = chrono::seconds(3600));
        \end{minted}
    \caption{Prototype of the function used to run the \gls{bw} algorithm on multiple observations in CuPAAL.}
    \label{lst:baum-welch-multiple-observations}
\end{listing}


The C++ bindings file is then compiled to a shared library, which can be imported into \Jajapy.
\Jajapy\ can call the $cupaal\_bw\_symbolic$ function, which will then call the \Cupaal\ implementation of the \gls{bw} algorithm.

We create a new function in \Jajapy, called \texttt{cupaal\_bw\_symbolic}, which is used to call the \Cupaal\ implementation of the \gls{bw} algorithm, as seen in~\autoref{lst:jajapy-bw-symbolic}.

This function is used to prepare the model parameters from \Jajapy\ so they are in the correct format for \Cupaal.
The preparation is done at lines 7 through 20; after this, the \Cupaal\ implementation is called at line 22, where the \texttt{cupaal\_bw\_symbolic} function is called with the prepared parameters, giving the \Cupaal\ model as a return value.

The values are then extracted from the \Cupaal\ model and assigned to the \Jajapy\ model, as seen in lines 23 through 25. where they are reshaped to be in line with \Jajapy.

\begin{listing*}
    \begin{minted}{python}
def _bw_symbolic(self, max_iteration = 100, epsilon = 1e-2, outputPath = "", resultPath = ""):
    try:
        import libcupaal_bindings
    except ModuleNotFoundError:
        print("Cannot find module")

    states = [str(i) for i in range (self.h.nb_states)]
    labels = list(set(self.h.labelling))
    observations = []
    for times, sequences in zip(self.training_set.times, self.training_set.sequences):
        for i in range(times):
            observations.append(list(sequences))
    initial_state = self.h.initial_state.tolist()
    transitions = self.h.matrix.flatten().tolist()
    emissions = zeros((len(labels), self.h.nb_states))
    for row in range(len(labels)):
        for col in range(self.h.nb_states):
            if self.h.labelling[col] == labels[row]:
                emissions[row][col] = 1
    emissions = emissions.flatten().tolist()

    cupaal_model = libcupaal_bindings.cupaal_bw_symbolic( states, labels, observations, initial_state, transitions, emissions, max_iteration, epsilon, outputPath, resultPath)
    self.h.initial_state = array(cupaal_model.initial_distribution)
    self.h.matrix = array(cupaal_model.transitions).reshape( self.h.nb_states, self.h.nb_states)
    self.h.emissions = array(cupaal_model.emissions).reshape( len(labels), self.h.nb_states)
    return self.h
      \end{minted}
    \caption{Jajapy's implementation of the \gls{bw} algorithm using CuPAAL.}
    \label{lst:jajapy-bw-symbolic}
\end{listing*}


The fit function in \Jajapy\ is modified to call the \texttt{\_bw\_symbolic} function when a new parameter called \texttt{symbolic} is set to true, as seen in~\autoref{lst:jajapy-fit-cupaal}.


\begin{listing}
    \begin{minted}{python}
# Some parameters have been removed for brevity
def fit(self, output_file: str, output_file_prism: str, epsilon: float, max_it: int, symbolic: bool):
    # Removed preparation and settings number of processes, for brevity
    if symbolic :
        return self._bw_symbolic(max_it, epsilon, output_file, output_file_prism)
    else:
        return self._bw(max_it, pp, epsilon, output_file, output_file_prism, verbose, stormpy_output, return_data)
      \end{minted}
    \caption{Jajapy's fit function, which calls the CuPAAL implementation of the \gls{bw} algorithm when symbolic is set to true.}
    \label{lst:jajapy-fit-cupaal}
\end{listing}

A check is made to see if the symbolic parameter is set to true at line 4. When the parameter is true, the \Jajapy\ model will call the~\autoref{lst:jajapy-bw-symbolic} function, which will then call the \Cupaal\ implementation of the \gls{bw} algorithm.

\subsection{Integration Discussion}\label{subsec:integration-discussion}
The integration of \Cupaal\ into \Jajapy\ has been successful, allowing us to leverage the \gls{bw} algorithm for parameter estimation for \glspl{hmm} and \glspl{mc}.

The decision to use pybind11 for creating bindings between C++ and Python has proven effective, as it allows us to call C++ functions from Python. However, further considerations should be made for the future to determine if this approach remains the best.

The exact implementation of the symbolic fit function in \Jajapy, shown in~\autoref{lst:jajapy-fit-cupaal}, is to be discussed with the \Jajapy\ creator, and some changes are expected to be made.

The integration of \Cupaal\ into \Jajapy\ has been successful, allowing us to leverage the \gls{bw} algorithm for parameter estimation for \glspl{hmm} and \glspl{mc}.
