\section{Experiments}\label{sec:experiments}
In this section, we present an evaluation comparing the performance of two implementations of the \gls{bw} algorithm: the original version from \Jajapy\ and the new symbolic implementation introduced in \Cupaal.

For this comparison, we use a \gls{mc} model, taken from the QComp benchmark set~\cite{hartmanns2019quantitative}, which is a collection of models used for quantitative verification and learning tasks.
The goal is to assess the scalability of the symbolic implementation and its performance in terms of runtime and accuracy.

We designed three experiments to evaluate the performance of the symbolic implementation of the \gls{bw} algorithm in \Cupaal:

\begin{itemize}
    \item \textbf{Accuracy} — Comparing the similarity of learned models in terms of log-likelihood and model checking.
    \item \textbf{Scalability} — Evaluating how the symbolic implementation scales with increasing model size.
    \item \textbf{Controlled initialization} — Evaluating the scalability of the symbolic implementation when initializing the model hypothesis with fewer unique values.
\end{itemize}

The experiments aim to answer the following research questions:
\begin{itemize}
    \item \textbf{Question 1}: How does runtime scale as model size increases for \Cupaal\ vs \Jajapy?
    \item \textbf{Question 2}: What is the relative estimation accuracy of the symbolic implementation in \Cupaal\ compared to the original recursive implementation in \Jajapy?
    \item \textbf{Question 3}: How much does an informed initialization accelerate \Cupaal?
\end{itemize}


The experiments are designed to provide insights into the performance and scalability of the symbolic implementation of the \gls{bw} algorithm in \Cupaal\ and to compare it with the original recursive implementation in \Jajapy.


\subsection{Model}
The model used is the leader sync model~\cite{IR90}, a \gls{dtmc} from the QComp benchmark set~\cite{hartmanns2019quantitative}.
It simulates a group of processors selecting a leader.
The model sizes range from 26 to 1050 states, determined by the number of processors and the maximum number a leader can choose during election: $\{26, 69, 147, 61, 274, 812, 141, 1050\}$.

The non-linear progression in model size arises because the QComp benchmark defines model variants using two parameters: the number of processors and the maximum number a leader can select to be elected.

We selected this model due to its scalability and interpretability. To make it suitable for learning, we extended it with additional labels.
The original model had only three labels, which was insufficient for training.
We added three new labels: reading, deciding, and elected which correspond to key phases in the leader election process.
The added labels and properties used for evaluation are shown in \autoref{lst:leader-sync-additions}.

\begin{listing}
    \begin{minted}{text}
    label "reading" = s1=1&s2=1&s3=1;
    label "deciding" = s1=2&s2=2&s3=2;
    label "elected" = s1=3&s2=3&s3=3;

    P>=1 [ F "elected" ]
    R{"num_rounds"}=? [ F "elected" ]
    \end{minted}
    \caption{Labels added to the leader sync model and properties checked.}
    \label{lst:leader-sync-additions}
\end{listing}

\subsection{Experimental Setup}
All experiments were conducted on the same machine (see Appendix~\ref{sec:machine_specs} for hardware and environment details).

The experimental steps are as follows:

\begin{enumerate}
    \item Load the \Prism\ model.
    \item Generate 100 observation sequences of different lenghts, for experiment 1 and 2 we have $\{25, 50, 100\}$, and for experiment 3 we have $\{25, 50, 75, 100\}$.
    \item Create a random initial \gls{mc} using \texttt{jajapy.MC\_random}.
    \item Run the \gls{bw} algorithm for up to 4 hours or until the change in log-likelihood is less than 0.01 (default stopping criterion in \Jajapy).
    \item Record runtime, number of iterations, log-likelihood, and save the resulting model.
\end{enumerate}

We save both the initial models and observations in files to ensure both implementations use the same inputs.
The generation of the training set and the randomization of the model are done using \Jajapy, which provides a convenient way to generate random models and training sets.
The training set is generated by the \gls{sul}, which is then used to train the randomized model.
Each configuration (model size and dataset size) is repeated ten times to compute average results.

We do not measure memory usage, as the symbolic implementation is implemented in C++ using Python bindings making it difficult to measure memory usage accurately, therefore, we focus on runtime and accuracy.


\subsection{Experiment 1: Accuracy}\label{sec:exp_accuracy}
This experiment evaluates the accuracy of \Cupaal\ compared to the original \Jajapy.
The goal is to measure the similarity of the symbolic implementation in \Cupaal\ to the original recursive implementation in \Jajapy\ accuracy-wise.
We compare the log-likelihood, which is a measure of how well the model fits the data in the symbolic implementation against the original recursive implementation.
We also measure the absolute error of model checking properties to the correct model, which is the leader sync model from the QComp benchmark set.
The properties can be seen in \autoref{lst:leader-sync-additions}, and the properties are taken from the QComp benchmark set.

The absolute error is defined as:
\[
    \text{Error} = |e - r|,
\]
where $e$ is the estimated transition probability and $r$ is the reference value from the original model.

\subsection{Experiment 2: Scalability}\label{sec:exp_scalability}
This experiment evaluates the scalability of the symbolic implementation of the \gls{bw} algorithm in \Cupaal.
The goal is to measure the runtime performance of the symbolic implementation as the size of the model increases.
The experiment measures the average runtime of the \gls{bw} algorithm for each model size and number of observation sequences.

\subsection{Experiment 3: Controlled initialization}\label{sec:exp_extra_scalability}
The third experiment evaluates the scalability of the symbolic implementation in \Cupaal\ when adjusting the initialization of the model hypothesis.

This experiment aims to measure the scalability of \Cupaal\ under circumstances that are theoretically good for the symbolic implementation.
The more repeated values the transition matrix contains, the sparser the \gls{add} representing it will be.
By initializing the transition matrix with a reduced amount of different values, we hope that the symbolic approach might benefit.

For the first experiment, the transition matrix was initialized randomly.
For this experiment, instead, we only use $|S|$ different values in the transition matrix.
It is expected that this improves the speed of each iteration of the \gls{bw} algorithm, as it reduces the number of unique computations necessary for the symbolic implementation.
