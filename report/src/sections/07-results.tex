\section{Results}\label{sec:results}
In this section, we present the results of our experiments comparing the performance of Jajapy and CuPAAL, using the Baum-Welch algorithm.
The metrics used for comparison are the time taken to train the model, the number of iterations needed, the average error, and the log-likelihood of the model.

The experiments were conducted on the same machine, the specifications of which are provided in Table\todo{ref to pc specs}.

\subsection{Time taken to train the model}\label{subsec:results_time}
The time taken to train the model is an important metric to consider when comparing the performance of Jajapy and CuPAAL.
The time taken to train the model is the time taken to complete the Baum-Welch algorithm, and the time is measured in seconds.

The results are displayed visually in figure\todo{ref to figure}.
The figure displays two bars for each model used in the experiment, one for Jajapy and one for CuPAAL.
The height of the bars represents the time taken to train the model, and the error bars represent the standard deviation of the time taken to train the model.

The results show a clear difference in the time taken to train the model between Jajapy and CuPAAL.
CuPAAL is significantly faster than Jajapy, acress all models used in the experiment.
The difference between the two tools, in terms of time taken to train the model, is especially pronounced for the larger models used in the experiment.


\subsection{Accuracy of the model}\label{subsec:results_accuracy}
The accuracy of the model is also an important metric to consider when comparing the performance of Jajapy and CuPAAL.
The accuracy of the model is measured using the log-likelihood of the model, which is a measure of how well the model fits the data.

The results are displayed visually in figure\todo{ref to figure}.
The figure displays two bars for each model used in the experiment, one for Jajapy and one for CuPAAL.
The height of the bars represents the log-likelihood of the model, and the error bars represent the standard deviation of the log-likelihood of the model.

Across all models used in the experiment, the results show that the accuracy of the model is similar for Jajapy and CuPAAL.

Which indicates that the difference in the time taken to train the model between Jajapy and CuPAAL does not come at the cost of accuracy.


\subsection{Number of iterations needed}\label{subsec:results_iterations}
The number of iterations needed to train the model is another important metric to consider when comparing the performance of Jajapy and CuPAAL.
The number of iterations needed is the number of iterations of the Baum-Welch algorithm needed to converge to a solution.

The results are displayed visually in table\todo{ref to table of results}.
The columns "Number of iterations" in the table, displays the number of iterations needed to train the model.
Along each row of the tables, is the model used in the experiment.

The results show that the number of iterations needed to train the model is similar for Jajapy and CuPAAL.

