\section{Preliminaries}\label{sec:preliminaries}
This section provides an overview of the theoretical background necessary to understand the rest of the report.
We begin by defining the key concepts of a \gls{hmm} and a \gls{mdp}, which are the two main models used in this report.


\subsection{Hidden Markov Model}\label{subsec:hmm}
\glspl{hmm} were introduced by Baum and Petrie in 1966~\cite{baum1966statistical} and have since been widely used in various fields, such as speech recognition~\cite{chavan2013overview}, bioinformatics~\cite{ciocchetta2009bio}, and finance~\cite{mamon2007hidden}.
\begin{definition}[Hidden Markov Model]
    A Hidden Markov Model (HMM) is a tuple $\mathcal{M} = (S, \mathcal{L}, \mathscr{l}, \tau,  \pi)$, where:
    \begin{itemize}
        \item $S$ is a finite set of states.
        \item $\mathcal{L}$ is a finite set of labels.
        \item $\mathscr{l}: S \rightarrow D(\mathcal{L})$ is the emission function.
        \item $\tau: S \rightarrow D(S)$ is the transition function.
        \item $\pi \in D(S)$ is the initial distribution.
    \end{itemize}
\end{definition}

$D(X)$ denotes the set of probability distributions over a  finite set $X$.
The emission function $\mathscr{l}$ describes the probability of emitting a label given a state.
The transition function $\tau$ describes the probability of transitioning from one state to another.
The initial distribution $\pi$ describes the probability of starting in a given state.
An \gls{hmm} is a statistical model that describes a system that evolves over time.
The system is assumed to hold the Markov property, meaning that the future state of the system only depends on the current state and not on the past states.
The system is also assumed to be unobservable, meaning that the states are hidden and cannot be directly observed.
Instead, the system emits observations, which are used to infer the hidden states.

An example of an \gls{hmm} is a weather model where the hidden state represents the actual weather (sunny, rainy, or cloudy), but we only observe indirect signals, such as whether someone is carrying an umbrella.

\subsection{Continuous Time Hidden Markov Model}\label{subsec:ctmc}
In the above definition, we have defined a discrete-time \gls{hmm}, meaning that the system evolves in discrete time steps.
In this report, we are interested in continuous-time systems, where the system evolves in continuous time.
To model continuous-time systems, we use a \gls{cthmm}, which is an extension of the \gls{hmm} to continuous time.
\begin{definition}[Continuous Time Hidden Markov Model]
    A Continuous Time Hidden Markov Model (CTHMM) is a tuple $\mathcal{M} = (S, \mathcal{L}, \mathscr{l}, \tau,  \pi, \lambda)$, where S, $\mathcal{L}$, $\mathscr{l}$, $\tau$, and $\pi$ are as defined in the \gls{hmm} definition.
    \begin{itemize}
        \item $\lambda: S \times S \rightarrow \mathbb{R}_{\geq 0}$ is the rate function.
    \end{itemize}
\end{definition}

The rate function $\lambda$ determines the transition rate between states, meaning that the time spent in a state before transitioning follows an exponential distribution.

\subsection{Markov Decision Process}\label{subsec:mdp}
\glspl{mdp} were introduced by Bellman in 1957~\cite{bellman1957} and have since been widely used in various fields, such as robotics~\cite{mdprobotics2020}, finance~\cite{mdpfinance2023}, and healthcare~\cite{mdphealthcare2021}.
\begin{definition}[Markov Decision Process]
    A Markov Decision Process (MDP) is a tuple $\mathcal{M} = (S, A, \mathcal{L}, \mathscr{l}, \tau, R, \gamma)$, where:
    \begin{itemize}
        \item $S$ is a finite set of states.
        \item $A$ is a finite set of actions.
        \item $\mathcal{L}$ is a finite set of labels.
              %\item $\mathscr{l}: S \rightarrow D(\mathcal{L})$ is the emission function.
        \item $\tau: S \times A \rightarrow D(S)$ is the transition function.
        \item $R: S \times A \rightarrow \mathbb{R}$ is the reward function.
        \item $\pi \in D(S)$ is the initial distribution.
    \end{itemize}
\end{definition}

A \gls{mdp} works by an agent interacting with an environment.
Unlike an \gls{hmm}, the agent  takes actions in the environment and receives rewards.
An agent is the decision-maker that interacts with the environment by taking actions.
The environment is modeled as a \gls{mdp}, which consists of a set of states $S$, a set of actions $A$, a set of labels $\mathcal{L}$.
Each state is directly observable, meaning the agent always knows which state it is in.
When the agent takes an action in a state, it transitions to a new state according to the transition function $\tau$.

The reward function $R$ describes the reward received when taking an action in a state.
The goal of an \gls{mdp} is to find a policy $\theta: S \rightarrow D(A)$ that maximizes the expected cumulative reward.
The policy $\theta$ describes the probability of taking an action given a state.

\begin{definition}[Policy]
    A policy $\theta$ is a function that maps states to actions, i.e., $\theta: S \rightarrow D(A)$.
\end{definition}

An example of an \gls{mdp} is a robot navigating a grid, where it can choose actions (move up, down, left, or right) and receives rewards based on reaching certain goal locations
